% Options for packages loaded elsewhere
\PassOptionsToPackage{unicode}{hyperref}
\PassOptionsToPackage{hyphens}{url}
%
\documentclass[
]{article}
\usepackage{amsmath,amssymb}
\usepackage{lmodern}
\usepackage{ifxetex,ifluatex}
\ifnum 0\ifxetex 1\fi\ifluatex 1\fi=0 % if pdftex
  \usepackage[T1]{fontenc}
  \usepackage[utf8]{inputenc}
  \usepackage{textcomp} % provide euro and other symbols
\else % if luatex or xetex
  \usepackage{unicode-math}
  \defaultfontfeatures{Scale=MatchLowercase}
  \defaultfontfeatures[\rmfamily]{Ligatures=TeX,Scale=1}
\fi
% Use upquote if available, for straight quotes in verbatim environments
\IfFileExists{upquote.sty}{\usepackage{upquote}}{}
\IfFileExists{microtype.sty}{% use microtype if available
  \usepackage[]{microtype}
  \UseMicrotypeSet[protrusion]{basicmath} % disable protrusion for tt fonts
}{}
\makeatletter
\@ifundefined{KOMAClassName}{% if non-KOMA class
  \IfFileExists{parskip.sty}{%
    \usepackage{parskip}
  }{% else
    \setlength{\parindent}{0pt}
    \setlength{\parskip}{6pt plus 2pt minus 1pt}}
}{% if KOMA class
  \KOMAoptions{parskip=half}}
\makeatother
\usepackage{xcolor}
\IfFileExists{xurl.sty}{\usepackage{xurl}}{} % add URL line breaks if available
\IfFileExists{bookmark.sty}{\usepackage{bookmark}}{\usepackage{hyperref}}
\hypersetup{
  pdftitle={Python Setup},
  pdfauthor={Alexander P. Christensen and Hudson Golino},
  hidelinks,
  pdfcreator={LaTeX via pandoc}}
\urlstyle{same} % disable monospaced font for URLs
\usepackage[margin=1in]{geometry}
\usepackage{color}
\usepackage{fancyvrb}
\newcommand{\VerbBar}{|}
\newcommand{\VERB}{\Verb[commandchars=\\\{\}]}
\DefineVerbatimEnvironment{Highlighting}{Verbatim}{commandchars=\\\{\}}
% Add ',fontsize=\small' for more characters per line
\usepackage{framed}
\definecolor{shadecolor}{RGB}{248,248,248}
\newenvironment{Shaded}{\begin{snugshade}}{\end{snugshade}}
\newcommand{\AlertTok}[1]{\textcolor[rgb]{0.94,0.16,0.16}{#1}}
\newcommand{\AnnotationTok}[1]{\textcolor[rgb]{0.56,0.35,0.01}{\textbf{\textit{#1}}}}
\newcommand{\AttributeTok}[1]{\textcolor[rgb]{0.77,0.63,0.00}{#1}}
\newcommand{\BaseNTok}[1]{\textcolor[rgb]{0.00,0.00,0.81}{#1}}
\newcommand{\BuiltInTok}[1]{#1}
\newcommand{\CharTok}[1]{\textcolor[rgb]{0.31,0.60,0.02}{#1}}
\newcommand{\CommentTok}[1]{\textcolor[rgb]{0.56,0.35,0.01}{\textit{#1}}}
\newcommand{\CommentVarTok}[1]{\textcolor[rgb]{0.56,0.35,0.01}{\textbf{\textit{#1}}}}
\newcommand{\ConstantTok}[1]{\textcolor[rgb]{0.00,0.00,0.00}{#1}}
\newcommand{\ControlFlowTok}[1]{\textcolor[rgb]{0.13,0.29,0.53}{\textbf{#1}}}
\newcommand{\DataTypeTok}[1]{\textcolor[rgb]{0.13,0.29,0.53}{#1}}
\newcommand{\DecValTok}[1]{\textcolor[rgb]{0.00,0.00,0.81}{#1}}
\newcommand{\DocumentationTok}[1]{\textcolor[rgb]{0.56,0.35,0.01}{\textbf{\textit{#1}}}}
\newcommand{\ErrorTok}[1]{\textcolor[rgb]{0.64,0.00,0.00}{\textbf{#1}}}
\newcommand{\ExtensionTok}[1]{#1}
\newcommand{\FloatTok}[1]{\textcolor[rgb]{0.00,0.00,0.81}{#1}}
\newcommand{\FunctionTok}[1]{\textcolor[rgb]{0.00,0.00,0.00}{#1}}
\newcommand{\ImportTok}[1]{#1}
\newcommand{\InformationTok}[1]{\textcolor[rgb]{0.56,0.35,0.01}{\textbf{\textit{#1}}}}
\newcommand{\KeywordTok}[1]{\textcolor[rgb]{0.13,0.29,0.53}{\textbf{#1}}}
\newcommand{\NormalTok}[1]{#1}
\newcommand{\OperatorTok}[1]{\textcolor[rgb]{0.81,0.36,0.00}{\textbf{#1}}}
\newcommand{\OtherTok}[1]{\textcolor[rgb]{0.56,0.35,0.01}{#1}}
\newcommand{\PreprocessorTok}[1]{\textcolor[rgb]{0.56,0.35,0.01}{\textit{#1}}}
\newcommand{\RegionMarkerTok}[1]{#1}
\newcommand{\SpecialCharTok}[1]{\textcolor[rgb]{0.00,0.00,0.00}{#1}}
\newcommand{\SpecialStringTok}[1]{\textcolor[rgb]{0.31,0.60,0.02}{#1}}
\newcommand{\StringTok}[1]{\textcolor[rgb]{0.31,0.60,0.02}{#1}}
\newcommand{\VariableTok}[1]{\textcolor[rgb]{0.00,0.00,0.00}{#1}}
\newcommand{\VerbatimStringTok}[1]{\textcolor[rgb]{0.31,0.60,0.02}{#1}}
\newcommand{\WarningTok}[1]{\textcolor[rgb]{0.56,0.35,0.01}{\textbf{\textit{#1}}}}
\usepackage{graphicx}
\makeatletter
\def\maxwidth{\ifdim\Gin@nat@width>\linewidth\linewidth\else\Gin@nat@width\fi}
\def\maxheight{\ifdim\Gin@nat@height>\textheight\textheight\else\Gin@nat@height\fi}
\makeatother
% Scale images if necessary, so that they will not overflow the page
% margins by default, and it is still possible to overwrite the defaults
% using explicit options in \includegraphics[width, height, ...]{}
\setkeys{Gin}{width=\maxwidth,height=\maxheight,keepaspectratio}
% Set default figure placement to htbp
\makeatletter
\def\fps@figure{htbp}
\makeatother
\setlength{\emergencystretch}{3em} % prevent overfull lines
\providecommand{\tightlist}{%
  \setlength{\itemsep}{0pt}\setlength{\parskip}{0pt}}
\setcounter{secnumdepth}{-\maxdimen} % remove section numbering
\usepackage[sfdefault]{cabin}
\usepackage{hyperref}
\hypersetup{
  colorlinks = true,
  allcolors = blue
}

\ifluatex
  \usepackage{selnolig}  % disable illegal ligatures
\fi

\title{Python Setup}
\author{Alexander P. Christensen and Hudson Golino}
\date{February 27, 2022}

\begin{document}
\maketitle

\hypertarget{getting-started-with-python}{%
\subsection{1. Getting Started with
Python}\label{getting-started-with-python}}

If you have Python already installed on your computer, then skip to
section 2 to check your Python installation.

The first step is to download Python. The recommended version is 3.9.7:
\url{https://www.python.org/downloads/release/python-397/}. Below are
the recommended settings when installing Python:

\begin{itemize}
\item
  Check the ``Add Python VERSION.NUMBER to PATH''
\item
  Then, select ``Customize installation''
\item
  Make sure all ``Optional Features'' are checked and click ``Next''
\item
  For ``Advanced Options,'' check ``Precompile standard library'' and
  click ``Install''
\end{itemize}

Feel free to choose a custom install location. The
\texttt{python\_setup()} function will automatically detect where the
path to your Python installation is. You can also enter this install
location manually.

After Python is finished installing, you should check whether it's been
installed properly.

\hypertarget{check-python-installation}{%
\subsection{2. Check Python
Installation}\label{check-python-installation}}

The following code will check if Python is installed and where Python is
installed:

\begin{Shaded}
\begin{Highlighting}[]
\FunctionTok{print}\NormalTok{(}\StringTok{"Is Python installed?"}\NormalTok{)}
\NormalTok{reticulate}\SpecialCharTok{::}\FunctionTok{py\_available}\NormalTok{(}\ConstantTok{TRUE}\NormalTok{)}
\end{Highlighting}
\end{Shaded}

\begin{verbatim}
[1] "Is Python installed?"
[1] TRUE
\end{verbatim}

If \texttt{FALSE}, then try the following steps:

\begin{itemize}
\item
  Restart your computer and try the above code again
\item
  If still \texttt{FALSE}, then go back to section 1 and try
  re-installing Python on your computer following the instructions
\end{itemize}

If \texttt{TRUE}, then Python has successfully installed on your
computer. To find out where, the following code can be used:

\begin{Shaded}
\begin{Highlighting}[]
\FunctionTok{print}\NormalTok{(}\StringTok{"Where is Python installed?"}\NormalTok{)}
\NormalTok{reticulate}\SpecialCharTok{::}\FunctionTok{py\_config}\NormalTok{()}
\end{Highlighting}
\end{Shaded}

\begin{verbatim}
[1] "Where is Python installed?"
python:         D:/anaconda3/python.exe
libpython:      D:/anaconda3/python39.dll
pythonhome:     D:/anaconda3
version:        3.9.7 (default, Sep 16 2021, 16:59:28) [MSC v.1916 64 bit (AMD64)]
Architecture:   64bit
numpy:          D:/anaconda3/Lib/site-packages/numpy
numpy_version:  1.20.3
\end{verbatim}

To make R aware of your Python installation, you can use the following
code:

\begin{Shaded}
\begin{Highlighting}[]
\FunctionTok{python\_setup}\NormalTok{()}
\NormalTok{[}\DecValTok{1}\NormalTok{] }\StringTok{"D:/anaconda3/python.exe"}
\end{Highlighting}
\end{Shaded}

\hypertarget{installing-the-transformers-module}{%
\subsection{\texorpdfstring{3. Installing the \texttt{transformers}
Module}{3. Installing the transformers Module}}\label{installing-the-transformers-module}}

With Python installed, the next step is to install the
\texttt{transformers} module. To do so, you'll need to open a command
line terminal on your computer. Once open, you can start by upgrading
\texttt{pip}:

\texttt{python\ -m\ pip\ install\ -\/-upgrade\ pip}

Before installing the \texttt{transformers} module, a few other packages
need to be installed: \texttt{PyTorch} and \texttt{Tensorflow}.

PyTorch can be installed by following the instructions on the website:
\url{https://pytorch.org/}. Select the ``Stable'' build and your
operating system. You will most likely use the ``Pip'' package for
install but ``Conda'' is another common package if you're using Anaconda
or miniconda (see \url{https://www.anaconda.com/}). Language should be
``Python'' and platform can be ``CPU.'' ``CUDA'' is used for GPU
processing but it is not necessary and requires additional steps for
setting up CUDA.

From the command line, here are the installs for different operating
systems:

Windows \texttt{pip\ install\ torch\ torchvision\ torchaudio}

Macs \texttt{pip\ install\ torch\ torchvision\ torchaudio}

Linux
\texttt{pip\ install\ torch==1.10.1+cpu\ torchvision==0.11.2+cpu\ torchaudio==0.10.1+cpu\ -f\ https://download.pytorch.org/whl/cpu/torch\_stable.html}

TensorFlow can be install using similar instructions
(\url{https://www.tensorflow.org/install}) but it is much more
straightforward. From the command line, you can use:

\texttt{pip\ install\ tensorflow}

Finally, you can install the \texttt{transformers} module:

\texttt{python\ -m\ pip\ install\ transformers}

To check that the \texttt{transformers} module was properly installed,
the following code can be run:

\begin{Shaded}
\begin{Highlighting}[]
\FunctionTok{print}\NormalTok{(}\StringTok{"\textquotesingle{}transformers\textquotesingle{} module installed?"}\NormalTok{)}
\NormalTok{reticulate}\SpecialCharTok{::}\FunctionTok{py\_module\_available}\NormalTok{(}\StringTok{"transformers"}\NormalTok{)}
\end{Highlighting}
\end{Shaded}

\begin{verbatim}
[1] "'transformers' module installed?"
[1] TRUE
\end{verbatim}

At this point, the \texttt{transformers} module should be ready-to-go on
your computer. If you're having trouble installing modules, then check
out this page:
\url{https://packaging.python.org/en/latest/tutorials/installing-packages/}.

\hypertarget{downloading-facebook-bart-large}{%
\subsection{4. Downloading Facebook BART
Large}\label{downloading-facebook-bart-large}}

The final step is to download the
\href{https://huggingface.co/facebook/bart-large-mnli}{Facebook's BART Large}
transformer model. The simplest way is to run our
\texttt{transformer\_scores} function with an example:

\begin{Shaded}
\begin{Highlighting}[]
\FunctionTok{transformer\_scores}\NormalTok{(}
    \AttributeTok{text =} \StringTok{"I like to go to parties"}\NormalTok{,}
    \AttributeTok{classes =} \FunctionTok{c}\NormalTok{(}\StringTok{"sociable"}\NormalTok{, }\StringTok{"warm"}\NormalTok{, }\StringTok{"assertive"}\NormalTok{, }\StringTok{"positive"}\NormalTok{),}
    \AttributeTok{multiple\_classes =} \ConstantTok{FALSE}\NormalTok{,}
    \AttributeTok{keep\_in\_env =} \ConstantTok{TRUE}
\NormalTok{)}
\end{Highlighting}
\end{Shaded}

\begin{Shaded}
\begin{Highlighting}[]
\SpecialCharTok{$}\StringTok{\textasciigrave{}}\AttributeTok{I like to go to parties}\StringTok{\textasciigrave{}}
\NormalTok{  sociable  assertive   positive       warm }
\FloatTok{0.81750941} \FloatTok{0.09567241} \FloatTok{0.04745624} \FloatTok{0.03936200}
\end{Highlighting}
\end{Shaded}

Compare to the \emph{emoxicon} package output

\begin{Shaded}
\begin{Highlighting}[]
\CommentTok{\# Load \textquotesingle{}emoxicon\textquotesingle{} package}
\FunctionTok{library}\NormalTok{(emoxicon)}

\CommentTok{\# Run emoxicon function}
\FunctionTok{emoxicon}\NormalTok{(}\StringTok{"I like to go to parties"}\NormalTok{)[}\SpecialCharTok{{-}}\FunctionTok{c}\NormalTok{(}\DecValTok{1}\NormalTok{,}\DecValTok{2}\NormalTok{)]}
\NormalTok{  AFRAID AMUSED ANGRY ANNOYED DONT\_CARE HAPPY INSPIRED SAD}
\DecValTok{1}      \DecValTok{0}      \DecValTok{0}     \DecValTok{0}       \DecValTok{0}         \DecValTok{0}     \DecValTok{0}        \DecValTok{2}   \DecValTok{0}
\end{Highlighting}
\end{Shaded}

The download will take some time: The model is 1.5Gb. The model will
download to your Python directory \emph{not} R. The model only needs to
be downloaded once and will be loaded each time the
\texttt{transformer\_scores} function is called. For repeated usage, the
\texttt{keep\_in\_env} argument should be set to \texttt{TRUE} (which is
the default) to keep the classifier loaded in the environment.

If you've made it this far, then you've successfully obtain sentiment
analysis scores from Facebook's BART Large transformer model. Go forth
and quantify the qualitative!

\end{document}
